\documentclass[a4paper,11pt]{article}

\usepackage{revy}
\usepackage[utf8]{inputenc}
\usepackage[T1]{fontenc}
\usepackage[danish]{babel}

\revyname{FysikRevy}
\revyyear{<%= year %>}
% HUSK AT OPDATERE VERSIONSNUMMER
\version{1.0}
\eta{$3$ minutter}
% Her skrives et estimat af sangens/sketchens varighed

\status{færdig/ideer}
% skriv færdig hvis sketchen er færdig, ellers skriv ideer

\title{<%= project_name %>}
\author{<%= user.first_name %> <%= user.last_name %>}

\begin{document}
\maketitle

%Liste over roller og deres indehavere:
\begin{roles}
	\role{R1}[Skuespiller 1] Rolle 1
	\role{R2}[Skuespiller 2] Rolle 2
	\role{MD}[Bjarke] Mogens Dam
	\role{H}[Gorm] Holger
\end{roles}

%Liste over rekvisitter. Behold teksten 'Person, der skaffer',
%indtil det er sikkert, hvem der skal have ansvaret for rekvisitten
\begin{props}
	\prop{Briks}[Person, der skaffer]
	\prop{2 lagener}[Person, der skaffer]
\end{props}

\begin{sketch}

\scene Lys op.
%Brug \scene inden alt scenespil inkl. lys og lyd fra TeXnikken.

\says{MD} Her på scenen er jeg Mogens Dam, men til dagligt er jeg Bjarke - ussel studerende.
%\says laver en ny replik. {MD} angiver hvilken rolle, replikken tilhører.

\says{H}[I skinger falset] Hvor sjovt! Jeg spiller Holger på scenen lige nu, men i virkeligheden er jeg Gorm.
%Beskrivelse af replikken i firkantet parantes

\says{MD} Det var da mærkeligt. \act{Klør sig i skægget}
%Brug \act om skuespilsting.

\scene Lys ned

\end{sketch}
\end{document}
