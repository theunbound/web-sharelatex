\documentclass[a4paper,11pt]{article}

\usepackage{revy}
\usepackage[utf8]{inputenc}
\usepackage[T1]{fontenc}
\usepackage[danish]{babel}

\revyname{FysikRevy}
\revyyear{<%= year %>}
\version{1.0}
\eta{$4$ minutter, $33$ sekunder}
\status{færdig/ideer}

\title{<%= project_name %>}
\melody{Originalens kunstner: ``Originalens titel''}
%angiver originalmelodien på formen 'kunstner: ``titel'' '

\begin{document}
\maketitle

\begin{roles}
	\role{S1}[Sanger 1] Sanger 1
	\role{S2}[Sanger 2] Sanger 2
	\role{HUM}[Bjarke] Humanist
	\role{F}[Emil] Fysikrevyst
\end{roles}
%Liste over roller og deres indehavere.

\begin{props}
	\prop{Briks}[Person, der skaffer]
	\prop{2 lagener}[Person, der skaffer]
\end{props}
%Liste over rekvisitter. Behold teksten [Person, der skaffer],
%indtil det er sikkert, hvem der skal have ansvaret for rekvisitten

\begin{song}
\scene Lys op, bandet spiller 'Anarchy in the UK' med Sex Pistols


\sings{HUM} Jeg er en humanist
Og jeg har ingen fez

\sings{F} Stop stop stop!!!
%Replikker laves også med \sing, når det er en sang man laver.

\scene Bandet stopper med at spille

\sings{F} Det rimer jo ikke engang, Bjarke.

\scene HUM ser slukøret ud.
%Skuespil angives også med \scene i sange.

\sings{F} Det er fandme ikke godt nok.

\sings{HUM} Piss...

\scene Lys ned
\end{song}

\end{document}